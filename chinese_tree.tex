% draw syntactic trees w/ Chinese characters

% use xelatex to compile!

\documentclass[11pt,a4paper]{article}

% for the tree
\usepackage{tikz}
\usepackage{tikz-qtree}
\usepackage{tikz-qtree-compat}
\usepackage{environ}
\makeatletter
\newsavebox{\measure@tikzpicture}
\NewEnviron{scaletikzpicturetowidth}[1]{%
  \def\tikz@width{#1}%
  \def\tikzscale{1}\begin{lrbox}{\measure@tikzpicture}%
  \BODY
  \end{lrbox}%
  \pgfmathparse{#1/\wd\measure@tikzpicture}%
  \edef\tikzscale{\pgfmathresult}%
  \BODY
}
\makeatother

% Chinese
%\usepackage[UTF8]{ctex}
\usepackage{xeCJK}
\setCJKmainfont{SimSun}

% for colored text
\newcommand{\red}[1]{\textcolor{red}{#1}}
\newcommand{\mybox}[1]{\framebox[1.02\width]{#1}}

% box around text
\usepackage{pbox}

\begin{document}

\begin{figure*}[t]
	\scalebox{0.70}{
		\begin{tikzpicture}
		\Tree [.\mybox{VP} [.Head:VC2 看 ] [.aspect:Di 著 ] [.\mybox{goal:NP} [.\mybox{predication:VP·的}\red{~(depth=0)} [.\mybox{head:VP}\red{~(1)} [.\mybox{location:NP}\red{~(2)} [.property:Nab\red{~(3)} 窗 ] [.Head:Ncda\red{~(3)} 外 ] ]  [.\mybox{standard:PP}\red{~(2)} [.Head:P58\red{~(3)} 隨 ] [.\mybox{DUMMY:NP}\red{~(3)} [.Head:Naa\red{~(4)} 風 ] ] ] [.Head:VA11\red{~(2)} 飄動 ] ] [.Head:DE\red{~(1)} 的 ] ] [.Head:Nab 樹枝 ] ] ] ] ] 
		\end{tikzpicture}
	}%
	\caption{Example Sinica tree, w/ RC depths (phrasal nodes in rectangle)}
\end{figure*}

\end{document}
